\documentclass[a4paper,10pt]{article}
\usepackage[utf8]{inputenc}
\usepackage[T1]{fontenc}
\usepackage[brazil]{babel}
\usepackage{indentfirst}

\title{Fórmulas - Simparticle}
\author{}


\begin{document}

\maketitle

\section{Partículas Elétricas}

	\subsection{Lei de Coulomb}
		
		\[
			\left|F\right| = k_e \frac{\left|q_1 q_2\right|}{r^2}
		\]
		onde:
		\begin{description}
		 \item [$F$] força resultante
		 \item [$q_1$] potencial elétrico da partícula $1$
		 \item [$q_2$] potencial elétrico da partícula $2$
		 \item [$r$] distância entre as partículas
		 \item [$k_e$] constante de Coulomb, definida por:
			\[
			 k_e = \frac{1}{4 \pi \epsilon_0 \epsilon}
			\]
			sendo:
			\begin{description}
			 \item [$\epsilon_0$] permissividade do espaço
			 \item [$\epsilon$] permissividade relativa do material que recebe a carga
			\end{description}
			Geralmente, usa-se:
			\[
			 k_e = 8.987 551 787 368 176 4 \cdot 10^9N m^2 C^{-2}
			\]


		\end{description}
		
		



\end{document}
