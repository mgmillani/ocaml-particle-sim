\documentclass[a4paper,10pt]{article}
\usepackage[utf8]{inputenc}
\usepackage[T1]{fontenc}
\usepackage[brazil]{babel}
\usepackage{hyperref}
\usepackage{indentfirst}
\usepackage{color}

\hypersetup
{
	colorlinks=true,
	linkcolor=blue
}

\title{Objetivos Atingidos}
\author{Aline \and Marcelo Garlet Millani}

\definecolor{notDone}{RGB}{240,30,5}
\definecolor{done}{RGB}{15,125,15}

\begin{document}

\maketitle

\section{}

\begin{itemize}
 \item Definição e uso de classes
		{\color{notDone}
		\begin{enumerate}
			\item Particle (Particle.ml)
			\item 
		\end{enumerate}
		}
 \item Encapsulamento e proteção dos atributos
 \item Organização do código em espaços de nome diferenciados
 \item Mecanismo de herança:
	\begin{itemize}
	 \item especificação de 3 níveis de hierarquia
		{\color{notDone}
		\begin{enumerate}
			\item Body (Body.ml)
			\item Particle (Particle.ml)
			\item Electric (Electric.ml)
		\end{enumerate}
		}
	 \item especificação de uma classe abstrata
		{\color{notDone}
		\begin{enumerate}
			\item Body (Body.ml)
		\end{enumerate}
		}
	 \item polimorfismo por inclusão
	\end{itemize}
 \item Polimorfismo paramétrico
	\begin{itemize}
	 \item especificação de algoritmo utilizando o recurso
	 \item especificação de estrutura de dados genérica
	 {\color{notDone}
		\begin{enumerate}
			\item Árvore para o algoritmo de Barnes-Hut (\url{http://en.wikipedia.org/wiki/Barnes-Hut})
		\end{enumerate}
		}
	\end{itemize}
 \item Polimorfismo por sobrecarga
 \item Especificação e uso de funções como elementos de primeira ordem
 \item Especificação e uso de funções de ordem maior
 {\color{notDone}
		\begin{enumerate}
			\item Body (Body.ml)
		\end{enumerate}
		}
 \item Uso de lista para manipulação de estruturas em funções de ordem maior (as funções devem ser puras)
 \item Uso de funções lambda
		{\color{notDone}
		\begin{enumerate}
			\item Poderia ser usado em uma função que aplica atrito (Physics.ml)
		\end{enumerate}
		}
 \item Currying
 \item Pattern matching
		{\color{done}
		\begin{enumerate}
			\item drawDots (main.ml)
			\item moveDots (Physics.ml)
		\end{enumerate}
		}
 \item Recursão como mecanismo de iteração
		{\color{done}
		\begin{enumerate}
			\item drawDots (main.ml)
			\item moveDots (Physics.ml)
		\end{enumerate}
		}
 \item Delegates


\end{itemize}
\end{document}
